%====================================================================================================
% FaultRecovery: Ampliação da Biblioteca de Tolerância a Falhas
%====================================================================================================
% Apresentação do Trabalho de Conclusão de Curso
%----------------------------------------------------------------------------------------------------
% Autor				: Cleiton Gonçalves de Almeida
% Orientador		: Kleber Kruger
% Instituição 		: UFMS - Universidade Federal do Mato Grosso do Sul
% Campus     		: Coxim
%----------------------------------------------------------------------------------------------------
% Arquivo			: introducao.tex
% Data de criação	: 10 de setembro de 2016
%====================================================================================================

%\section{Referências Bibliográficas} \label{Sec:Referencias}

%\begin{frame}
%	\frametitle{Referências Bibliográficas}
%	\addcontentsline{toc}{chapter}{Referências Bibliográficas}
%	%\bibliographystyle{abnt}
%	\bibliographystyle{ieeetr} % Ordena por ordem de aparição.  
%	%\bibliographystyle{abbr} % Ordena por ordem alfabetica com nomes abreviados.
%	%\bibliographystyle{plain} % Ordena por ordem alfabetica com nomes por extenso.
%	\bibliography{bibliografia} % commented if *.bbl file included.
%\end{frame}

%\begin{frame}
%	\frametitle{Referências Bibliográficas}
%	\footnotesize
%	{
%		\begin{thebibliography}{99}	
%			\bibitem[Weber, 2003]{Weber:2003} Weber, T. S. (2003)
%			\newblock Tolerância a falhas: conceitos e exemplos.
%			\newblock \emph{Intech Brasil. Distrito 4 da ISA (The Instrumentation, System and Automation Society)}, 32 -- 42.
%		\end{thebibliography}
%	}
%\end{frame}
