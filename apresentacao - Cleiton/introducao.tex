%====================================================================================================
% FaultRecovery: Ampliação da Biblioteca de Tolerância a Falhas
%====================================================================================================
% Apresentação do Trabalho de Conclusão de Curso
%----------------------------------------------------------------------------------------------------
% Autor				: Cleiton Gonçalves de Almeida
% Orientador		: Kleber Kruger
% Instituição 		: UFMS - Universidade Federal do Mato Grosso do Sul
% Campus     		: Coxim
%----------------------------------------------------------------------------------------------------
% Arquivo			: introducao.tex
% Data de criação	: 10 de setembro de 2016
%====================================================================================================

\section{Introdução} \label{Sec:Introducao}

\begin{frame}
	\frametitle{Introdução}
	\begin{itemize}
		\item Os sistemas embarcados (\textit{embedded computers} ou \textit{embedded systems}) correspondem a maior classe de computadores e abrangem uma grande faixa de aplicações e desempenhos;
		\item Com a expansão da computação ubíqua, os sistemas embarcados estão cada vez mais presentes no cotidiano das pessoas.
		\item É importante que esses sistemas tolerem falhas, pois defeitos nessas aplicações podem causar transtornos e prejuízos.		
		\end{itemize}
\end{frame}

\subsection{Justificativa}

\begin{frame}
	\frametitle{Justificativa}
	\begin{itemize}		
		\item Falhas podem ser causadas por:
		\begin{itemize}
			\item \textit{Bugs} de software;
			
			\item Questões ambientais;
			\begin{itemize}
				\item Interferências eletromagnéticas;
				\item Pulsos transitórios causados por prótons, íons pesadores e elétrons; Que podem causar um \textit{bit-flip}
			\end{itemize}			
			\item Problemas intrínsecos;
			\begin{itemize}
				\item Desgaste (envelhecimento) dos componentes de \textit{hardware};				
				\item Problemas de fabricação;
			\end{itemize}
		\end{itemize}
	\end{itemize}
\end{frame}

\subsection{Objetivos}

\begin{frame}
	\frametitle{Objetivo}
	\begin{itemize}
		\item Objetivo geral: Ampliar o injetor de falhas (FaultInjector ) e a biblioteca de recuperação de falhas (FaultRecovery) desenvolvidos por Kruger em sua dissertação
		de mestrado.
	\end{itemize}
\end{frame}

