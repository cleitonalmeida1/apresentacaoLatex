%====================================================================================================
% FaultRecovery: Ampliação da Biblioteca de Tolerância a Falhas
%====================================================================================================
% Apresentação do Trabalho de Conclusão de Curso
%----------------------------------------------------------------------------------------------------
% Autor				: Cleiton Gonçalves de Almeida
% Orientador		: Kleber Kruger
% Instituição 		: UFMS - Universidade Federal do Mato Grosso do Sul
% Campus     		: Coxim
%----------------------------------------------------------------------------------------------------
% Arquivo			: introducao.tex
% Data de criação	: 10 de setembro de 2016
%====================================================================================================

\section{Conclusão} \label{Sec:Conclusao}

\begin{frame}
	\frametitle{Conclusão}
	\begin{itemize}
		\item Sem a FaultRecovery:
		\begin{itemize}			
			\item O \textit{firmware} falhou e não continuou a sequência da máquina de estados.
		\end{itemize}
		\item Com a FaultRecovery:
		\begin{itemize}
			\item O \textit{firmware} falhou e continuou a sequência da máquina de estados de acordo com o ponto de recuperação configurado.	
			\item Aumento no tempo de execução do \textit{firmware}. No entanto a maioria das aplicações embarcadas não tem como fator principal o tempo de execução, exceto algumas aplicações de tempo real.
			
		\end{itemize}
	\end{itemize}
\end{frame}

\begin{frame}
	\frametitle{Conclusão}
	\begin{itemize}
		\item Sem redundância de dados:
		\begin{itemize}			
			\item Nenhuma falha foi tolerada.
			\item Num total de 100 testes, 44\% falharam.
		\end{itemize}
		\item Com redundância de dados:
		\begin{itemize}
			\item 100\% das falhas foram toleradas. No entanto adiciona um custo de desempenho no tempo de processamento. 
			\item Num total de 100 testes, nenhum teste falhou.
		\end{itemize}
	\end{itemize}
\end{frame}

\begin{frame}
	\frametitle{Contribuições deste trabalho}
	\begin{itemize}
		\item Parte da biblioteca FaultRecovery está sendo utilizada pelo Projeto de extensão Coxim robótica sediado na UFMS - campus Coxim para implementar uma máquina de estados para um carrinho seguidor de linha.
	\end{itemize}
\end{frame}

\begin{frame}
	\frametitle{Trabalhos Futuros}
	\begin{itemize}
		\item Testar o mapeamento de memória incluindo na biblioteca FaultInjector para outros modelos além do modelo \textit{mbed} LPC1768.
		\item Salvar as cópias utilizadas pela classe TData, para se ter redundância de dados, em outras regiões de memória.
		\item Aperfeiçoar a biblioteca FaultRecovery e a classe TData para diminuir o tempo de processamento em uma aplicação embarcada.
		\item Explorar a injeção de falhas na memória flash, implementando um \textit{firmware} e injetando falhas enquanto ele está em execução.
	\end{itemize}
\end{frame}

\begin{frame}
	\frametitle{Trabalhos Futuros}
		\centering \resizebox{!}{2cm}{Obrigado!}
\end{frame}

